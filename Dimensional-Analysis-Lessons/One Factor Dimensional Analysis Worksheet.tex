% !TEX TS-program = pdflatex
% !TEX encoding = UTF-8 Unicode

% This is a simple template for a LaTeX document using the "article" class.
% See "book", "report", "letter" for other types of document.

\documentclass[12pt]{article} % use larger type; default would be 10pt

\usepackage[utf8]{inputenc} % set input encoding (not needed with XeLaTeX)

%%% Examples of Article customizations
% These packages are optional, depending whether you want the features they provide.
% See the LaTeX Companion or other references for full information.

%%% PAGE DIMENSIONS
\usepackage{geometry} % to change the page dimensions
\geometry{letterpaper} % or letterpaper (US) or a5paper or....
 \geometry{margin=1in} % for example, change the margins to 2 inches all round
% \geometry{landscape} % set up the page for landscape
%   read geometry.pdf for detailed page layout information

\usepackage{graphicx} % support the \includegraphics command and options

% \usepackage[parfill]{parskip} % Activate to begin paragraphs with an empty line rather than an indent

%%% PACKAGES
\usepackage{booktabs} % for much better looking tables
\usepackage{array} % for better arrays (eg matrices) in maths
\usepackage{paralist} % very flexible & customisable lists (eg. enumerate/itemize, etc.)
\usepackage{verbatim} % adds environment for commenting out blocks of text & for better verbatim
\usepackage{subfig} % make it possible to include more than one captioned figure/table in a single float
\usepackage[nofiglist, notablist, nomarkers]{endfloat}
\usepackage{longtable}

\usepackage{titling}
% These packages are all incorporated in the memoir class to one degree or another...

%%% HEADERS & FOOTERS
\usepackage{fancyhdr} % This should be set AFTER setting up the page geometry
\pagestyle{fancy} % options: empty , plain , fancy
\renewcommand{\headrulewidth}{0pt} % customise the layout...
\lhead{Dimensional Analysis Practice Problems}\chead{}\rhead{Name:\hspace{1in}}
\lfoot{}\cfoot{}\rfoot{}

%%% SECTION TITLE APPEARANCE
\usepackage{sectsty}
\allsectionsfont{\sffamily\mdseries\upshape} % (See the fntguide.pdf for font help)
% (This matches ConTeXt defaults)

%%% ToC (table of contents) APPEARANCE
\usepackage[nottoc,notlof,notlot]{tocbibind} % Put the bibliography in the ToC
\usepackage[titles,subfigure]{tocloft} % Alter the style of the Table of Contents
\renewcommand{\cftsecfont}{\rmfamily\mdseries\upshape}
\renewcommand{\cftsecpagefont}{\rmfamily\mdseries\upshape} % No bold!

%%% END Article customizations

%%% The "real" document content comes below...
\pretitle{\begin{flushleft}\Large}
\setlength{\droptitle}{-3.5cm}
\title{Dimensional Analysis Practice}
\posttitle{\par\end{flushleft}}
\date{}

\begin{document}

\maketitle

\thispagestyle{empty}
\vspace{-2.5cm}

\begin{enumerate}
\item Using the steps below, show how to convert 12.5 km to feet. 

\begin{enumerate}[A.]
\item Fill in the blank space to make these statements true \\
 \underline{\hspace{2cm}} km =  \underline{\hspace{2cm}}  m \\
 \underline{\hspace{2cm}} m =  \underline{\hspace{2cm}}  cm \\
 \underline{\hspace{2cm}} cm =  \underline{\hspace{2cm}}  in \\
 \underline{\hspace{2cm}} in =  \underline{\hspace{2cm}}  ft \\

\item Write the measurement and units that you are starting with: 
\vspace{1cm}
\item Create you first conversion factor by dividing one of the true statements above. Show your method: 
\vspace{1cm}
\item Multiply your initial quantity by the first conversion factor. Cancel the appropriate units. 
\vspace{1cm}
\item Create your second conversion factor by dividing one the true statements above. Show your method: 
\vspace{1cm}
\item Multiply your answer to D by the second conversion factor. Cancel the appropriate units. 
\vspace{1cm}
\item Create your third conversion factor by dividing one the true statements above. Show your method: 
\vspace{1cm}
\item Multiply your answer to F by the third conversion factor. Cancel the appropriate units. \vspace{1cm}
\item Create your fourth conversion factor by dividing one the true statements above. Show your method: 
\vspace{1cm}
\item Multiply your answer to H by the third conversion factor. Cancel the appropriate units. 
\vspace{1cm}
\item Check: does your answer to J have the correct units? If so, what is your correct answer? 
\end{enumerate}


\clearpage
Using that same method on the front of this page, solve the following problems. Make sure that you show all the steps involved. You may attach extra notebook paper if necessary. You may need to look up definitions of units on wikipedia or google in order to write your series of true statements.


\item 3 miles to km 
\vspace{1.5in}
\item 2.7 lbs to kg
\vspace{1.5in}
\item 3 days to seconds
\vspace{1.5in}
\item 2.7 liters to gallons
\vspace{1.5in}
\item 135 lbs to grams
\vspace{1.5in}

\end{enumerate}
\end{document}
